\documentclass[12pt,reqno,final,pdftex]{amsart}\usepackage[]{graphicx}\usepackage[]{color}
%% maxwidth is the original width if it is less than linewidth
%% otherwise use linewidth (to make sure the graphics do not exceed the margin)
\makeatletter
\def\maxwidth{ %
  \ifdim\Gin@nat@width>\linewidth
    \linewidth
  \else
    \Gin@nat@width
  \fi
}
\makeatother

\definecolor{fgcolor}{rgb}{0.345, 0.345, 0.345}
\newcommand{\hlnum}[1]{\textcolor[rgb]{0.686,0.059,0.569}{#1}}%
\newcommand{\hlstr}[1]{\textcolor[rgb]{0.192,0.494,0.8}{#1}}%
\newcommand{\hlcom}[1]{\textcolor[rgb]{0.678,0.584,0.686}{\textit{#1}}}%
\newcommand{\hlopt}[1]{\textcolor[rgb]{0,0,0}{#1}}%
\newcommand{\hlstd}[1]{\textcolor[rgb]{0.345,0.345,0.345}{#1}}%
\newcommand{\hlkwa}[1]{\textcolor[rgb]{0.161,0.373,0.58}{\textbf{#1}}}%
\newcommand{\hlkwb}[1]{\textcolor[rgb]{0.69,0.353,0.396}{#1}}%
\newcommand{\hlkwc}[1]{\textcolor[rgb]{0.333,0.667,0.333}{#1}}%
\newcommand{\hlkwd}[1]{\textcolor[rgb]{0.737,0.353,0.396}{\textbf{#1}}}%
\let\hlipl\hlkwb

\usepackage{framed}
\makeatletter
\newenvironment{kframe}{%
 \def\at@end@of@kframe{}%
 \ifinner\ifhmode%
  \def\at@end@of@kframe{\end{minipage}}%
  \begin{minipage}{\columnwidth}%
 \fi\fi%
 \def\FrameCommand##1{\hskip\@totalleftmargin \hskip-\fboxsep
 \colorbox{shadecolor}{##1}\hskip-\fboxsep
     % There is no \\@totalrightmargin, so:
     \hskip-\linewidth \hskip-\@totalleftmargin \hskip\columnwidth}%
 \MakeFramed {\advance\hsize-\width
   \@totalleftmargin\z@ \linewidth\hsize
   \@setminipage}}%
 {\par\unskip\endMakeFramed%
 \at@end@of@kframe}
\makeatother

\definecolor{shadecolor}{rgb}{.97, .97, .97}
\definecolor{messagecolor}{rgb}{0, 0, 0}
\definecolor{warningcolor}{rgb}{1, 0, 1}
\definecolor{errorcolor}{rgb}{1, 0, 0}
\newenvironment{knitrout}{}{} % an empty environment to be redefined in TeX

\usepackage{alltt}
%% DO NOT DELETE OR CHANGE THE FOLLOWING TWO LINES!
%% $Revision$
%% $Date$
\usepackage[round,sort,elide]{natbib}
\usepackage{graphicx}
\usepackage{times}
\usepackage{rotating}
\usepackage{subfig}
\usepackage{color}
\newcommand{\aak}[1]{\textcolor{cyan}{#1}}
\newcommand{\mab}[1]{\textcolor{red}{#1}}
\newcommand{\cec}[1]{\textcolor{blue}{#1}}

\setlength{\textwidth}{6.25in}
\setlength{\textheight}{8.75in}
\setlength{\evensidemargin}{0in}
\setlength{\oddsidemargin}{0in}
\setlength{\topmargin}{-.35in}
\setlength{\parskip}{.1in}
\setlength{\parindent}{0.3in}

%% cleveref must be last loaded package
\usepackage[sort&compress]{cleveref}
\newcommand{\crefrangeconjunction}{--}
\crefname{figure}{Fig.}{Figs.}
\Crefname{figure}{Fig.}{Figs.}
\crefname{table}{Table}{Tables}
\Crefname{table}{Tab.}{Tables}
\crefname{equation}{Eq.}{Eqs.}
\Crefname{equation}{Eq.}{Eqs.}
\crefname{appendix}{Appendix}{Appendices}
\Crefname{appendix}{Appendix}{Appendices}
\creflabelformat{equation}{#2#1#3}

\theoremstyle{plain}
\newtheorem{thm}{Theorem}
\newtheorem{corol}[thm]{Corollary}
\newtheorem{prop}[thm]{Proposition}
\newtheorem{lemma}[thm]{Lemma}
\newtheorem{defn}[thm]{Definition}
\newtheorem{hyp}[thm]{Hypothesis}
\newtheorem{example}[thm]{Example}
\newtheorem{conj}[thm]{Conjecture}
\newtheorem{algorithm}[thm]{Algorithm}
\newtheorem{remark}{Remark}
\renewcommand\thethm{\arabic{thm}}
\renewcommand{\theremark}{}

\numberwithin{equation}{part}
\renewcommand\theequation{\arabic{equation}}
\renewcommand\thesection{\arabic{section}}
\renewcommand\thesubsection{\thesection.\arabic{subsection}}
\renewcommand\thefigure{\arabic{figure}}
\renewcommand\thetable{\arabic{table}}
\renewcommand\thefootnote{\arabic{footnote}}

\newcommand\scinot[2]{$#1 \times 10^{#2}$}
\newcommand{\code}[1]{\texttt{#1}}
\newcommand{\pkg}[1]{\textsf{#1}}
\newcommand{\dlta}[1]{{\Delta}{#1}}
\newcommand{\Prob}[1]{\mathbb{P}\left[#1\right]}
\newcommand{\Expect}[1]{\mathbb{E}\left[#1\right]}
\newcommand{\Var}[1]{\mathrm{Var}\left[#1\right]}
\newcommand{\dd}[1]{\mathrm{d}{#1}}
\newcommand{\citetpos}[1]{\citeauthor{#1}'s \citeyearpar{#1}}
\IfFileExists{upquote.sty}{\usepackage{upquote}}{}
\begin{document}



\section*{Consumer-Resource Model}

In the absence of parasitism, we assume that the consumer-resource interaction between the host ($S$) and its resource ($R$) can be modeled using the Rosenzweig-MacArthur (RM) model:

\begin{align}
\begin{split}
\frac{dR(t)}{dt} &= r R(t) \left(1 - \frac{R(t)}{K}\right) - \frac{f_S R(t) S(t)}{h + R(t)}, \\
\frac{dS(t)}{dt} &= \frac{\epsilon_S f_S R(t) S(t)}{h + R(t)} - d S(t).
\end{split}
\end{align}

The cogent feature of this model is its ability to produce either a stable equilibrium or stable limit cycles, depending on parameter values.
In particular, as the carrying capacity of the resource, $K$, is increased, the system can transition from equilibrium to limit cycles, a result termed the ``paradox of enrichment.''

\section*{Consumer-Resource-Parasite Model}
Here, we study the evolution of parasite virulence in such a consumer-resource system.
We are interested in two questions, in particular.
First, under what biological assumptions does the evolutionarily stable virulence depend on host resources?
Second, how is the evolutionarily stable virulence affected by consumer-resource dynamics?

To begin, we augment the RM model by dividing the host population into susceptible ($S$) and infected ($Q$) classes, with transmission between classes dependent on contact between susceptible and infected classes.
We assume that both susceptible and infected class forage at the same rate, and that all transmission is horizontal, so hosts are born susceptible.
We also assume that infected hosts suffer additional mortality due to parasites at the per-capita rate $v$, so that $v$ is the virulence of the parasite.
We assume a classic virulence-transmission trade-off, such that the transmission rate, $\beta$, is a function of $v$.
Under these assumptions, the model becomes:
\begin{align}\label{RSQ}
\begin{split}
\frac{dR(t)}{dt} &= r R(t) \left(1 - \frac{R(t)}{K}\right) - \frac{f_S R(t) (S(t)+Q(t))}{h + R(t)}, \\
\frac{dS(t)}{dt} &= \frac{\epsilon_S f_S R(t) (S(t)+Q(t))}{h + R(t)} - \beta(v) S(t) Q(t)- d S(t), \\
\frac{dQ(t)}{dt} &= \beta(v) S(t) Q(t) - (d + v)Q(t).
\end{split}
\end{align}

This model retains the dynamical properties of the standard RM model, including that increasing the value of $K$ will destabilize the dynamics.
The additional mortality imposed by the parasite, however, has a positive effect on the stability of the system, in that the system becomes unstable at larger values of $K$.

\section*{Virulence evolution}
To study the evolution of virulence, we take an adaptive dynamics approach.
To model \ref{RSQ} above, we add a second strain of parasite (the 'mutant', $Q_m$)  that is assumed to differ from the 'resident' strain in its virulence ($v_m$ for the mutant, compared to $v$ for the resident).
The model then becomes:
\begin{align}\label{RSQQm}
\begin{split}
\frac{dR(t)}{dt} &= r R(t) \left(1 - \frac{R(t)}{K}\right) - \frac{f_S R(t) (S(t)+Q(t)+Q_m(t))}{h + R(t)}, \\
\frac{dS(t)}{dt} &= \frac{\epsilon_S f_S R(t) (S(t)+Q(t)+Q_m(t))}{h + R(t)} - \beta(v) S(t) Q(t) - \beta(v_m) S(t) Q_m(t)- d S(t), \\
\frac{dQ(t)}{dt} &= \beta(v) S(t) Q(t) - (d + v)Q(t), \\
\frac{dQ_m(t)}{dt} &= \beta(v_m) S(t) Q_m(t) - (d + v_m)Q_m(t).
\end{split}
\end{align}
Whether the mutant can invade the system is determined by the Jacobian matrix, lineared about a zero-mutant state ($Q_m(t) = 0$), since the mutant is assumed to rare.
\begin{equation}
J = \begin{pmatrix}
\frac{\partial}{\partial R}\left(\frac{dR}{dt}\right) & \frac{\partial}{\partial S}\left(\frac{dR}{dt}\right) & \frac{\partial}{\partial Q}\left(\frac{dR}{dt}\right) & \frac{\partial}{\partial Q_m}\left(\frac{dR}{dt}\right) \\
\frac{\partial}{\partial R}\left(\frac{dS}{dt}\right) & \frac{\partial}{\partial S}\left(\frac{dS}{dt}\right) & \frac{\partial}{\partial Q}\left(\frac{dS}{dt}\right) & \frac{\partial}{\partial Q_m}\left(\frac{dS}{dt}\right) \\
\frac{\partial}{\partial R}\left(\frac{dQ}{dt}\right) & \frac{\partial}{\partial S}\left(\frac{dQ}{dt}\right) & \frac{\partial}{\partial Q}\left(\frac{dQ}{dt}\right) & 0 \\
0 & 0 & 0 & \frac{\partial}{\partial Q_m}\left(\frac{dQ_m}{dt}\right)
\end{pmatrix}
\end{equation}
Notice that this matrix is upper block triangular, meaning that the stability properties of the full system can be determined by studying the 3x3 submatrix on the diagonal and the equation $\frac{\partial}{\partial Q_m}\left(\frac{dQ_m}{dt}\right)$.
The upper left 3x3 submatrix determines the stability of the $R-S-Q$ system when the mutant strain is absent.
In particular, if $R-S-Q$ system reaches a stable equilibrium $(R=\hat{R},S=\hat{S},Q=\hat{Q})$, then the mutant can invade if,
\begin{equation}
\frac{\partial}{\partial Q_m}\left(\frac{dQ_m}{dt}\right)|_{(R=\hat{R},S=\hat{S},Q=\hat{Q})} = \beta(v_m) \hat{S} - (d+v_m) > 0.
\end{equation}
If, on the other hand, the mutant-free system reaches a limit cycle ($(R=\hat{R(t)},S=\hat{S(t)},Q=\hat{Q(t)})$) of length $T$, then the mutant can invade if,
\begin{equation}
\frac{\partial}{\partial Q_m}\left(\frac{dQ_m}{dt}\right)|_{(R=\hat{R(t)},S=\hat{S(t)},Q=\hat{Q(t)})} = \beta(v_m) \frac{1}{T} \int_{0}^T \left(\hat{S(t)} dt\right) - (d+v_m) > 0.
\end{equation}

\subsection*{Equilibrium dynamics}
If the mutant-free system approaches an equilibrium, then the mutant can invade if
\begin{equation}
r = \beta(v_m) \hat{S} - (d+v_m) > 0.
\end{equation}
We term $r$ the 'invasion fitness' of the mutant strain.

By setting $dQ/dt = 0$, we find that, at equilibrium, $\hat{S} = (d+v)/\beta(v)$, which we can plug into the invasion fitness expression.
We seek to use this expression to find a value $v = v^*$ such that $r \leq 0$ for all $v_m \neq v^*$, termed the 'singular strategy.'
This singular strategy is any value $v = v^*$ such that,
\begin{equation}
\frac{dr}{dv_m}|_{v_m=v} = 0.
\end{equation}
Plugging the equilibrium for $\hat{S}$ into $r$, a singular strategy will therefore satisfy $(d+v^*)\beta'(v^*)-\beta(v^*)=0$.

Whether such a singular strategy is represents a fitness maximum, and is thus a true evolutionarily stable strategy, requires
\begin{equation}
\frac{d^2r}{dv_m^2}|_{v_m=v=v^*} = \frac{(d+v^*) \beta''(v^*)}{\beta(v^*)} \leq 0.
\end{equation}
Thus, if transmission rate is a saturating function of virulence, then any singular strategy will also be evolutionarily stable.

Whether this ESS virulence can actually be reached by a sequence of small mutational events is determined by the convergence stability condition
\begin{equation}
\frac{d}{dv}\left(\frac{dr}{dv_m}|_{v_m=v}\right)|_{v=v^*} = \frac{\beta''(v^*) \beta(v^*) (d+v^*) - \beta'(v^*)\left((d+v^*)\beta'(v^*)-\beta(v^*)\right)}{\beta(v^*)^2} < 0.
\end{equation}
Note that, since we are evaluating this derivative at the singular strategy, $(d+v^*)\beta'(v^*)-\beta(v^*) = 0$, so the sign of the convergence stability condition is identical to the evolutionarily stability condition, and depends entirely on the sign of $\beta''(v^*)$.
Thus, if transmission rate is a saturating function of virulence, then any evolutionarily stable strategy will also be convergence stable.

For example, if $\beta(v) = \beta_0 \frac{v}{1+v}$, then the ESS and CSS singular strategy is $v^* = \sqrt{d}$





\end{document}


